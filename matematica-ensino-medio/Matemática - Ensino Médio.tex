\documentclass[12pt,a4paper]{book}
\usepackage[utf8]{inputenc}
\usepackage[portuguese]{babel}
\usepackage[T1]{fontenc}
\usepackage{amsmath}
\usepackage{amsfonts}
\usepackage{amssymb}
\usepackage{makeidx}
\usepackage{graphicx}
\usepackage{booktabs}
\usepackage{color}
\usepackage{hyperref}
\usepackage{float}
\graphicspath{{./Figuras/}}    
\usepackage{multicol}
\definecolor{shadecolor}{rgb}{0.8,0.8,0.8}
\usepackage{amssymb} %Símbolos de Conjuntos numéricos
\usepackage{multicol} %Várias Colunas
\usepackage[left=3cm,right=2cm,top=3cm,bottom=2cm]{geometry}
\author{Leandro Vieira}
\title{Notas de Aula - Matemática no Ensino Médio}

\begin{document}
\maketitle
\tableofcontents

\chapter{Considerações Iniciais}

%--------------------------------------------------------------------------------------------------------------------------------
%--------------------------------------------------------------------------------------------------------------------------------

%\part{1ª Série do Ensino Médio}

%\part*{I Unidade}

%\part*{II Unidade}

%\part*{III Unidade}

%\part*{IV Unidade}

%--------------------------------------------------------------------------------------------------------------------------------
%--------------------------------------------------------------------------------------------------------------------------------
\part{2ª Série do Ensino Médio}

\part*{I Unidade}

\chapter{Conjuntos Numéricos}

	\section{Os Conjuntos Numéricos}
	\section{A Reta Real}
	\section{O Plano Cartesinao}
		\subsection{Equação da Reta}
		
\chapter{Matrizes e Determinantes}

	
	\section{Matrizes}
		\subsection{Introdução}
		\subsection{Construção}
		\subsection{Operações com Matrizes}
		\subsection{Multiplicação de Matrizes}
	
	\section{Determinantes}
		\subsection{Introdução}
		\subsection{Matrizes $2\times 2$}
		\subsection{Matrizes $3\times 3$}
		
	\section{Aplicações de Determinantes}
		\subsection{Sistemas Lineares: 2 equações e 2 icógnitas}
		\subsection{Sistemas Lineares: 3 equações e 3 icógnitas}
		\subsection{Área de Triângulos no Plano}
		\subsection{Equação da Reta no Plano}
		
\chapter{Funções Afim}

	\section{Introdução ao Estudo de Funções}	
	\section{Funçõe Afim}
		\subsection{Introdução}
		\subsection{Gráficos}
		\subsection{Progressões Aritméticas}
		\subsection{A Função Afim na Física}

\chapter{Áreas e Volumes}

	\section{Unidades de Medida}
	\section{Áreas}
	\section{Volumes}		
		
\part*{II Unidade}
	\chapter{Funções Quadráticas}
	
		\section{Introdução}
		\section{Equações do Segundo Grau}
		\section{A Função Quadrática}
		\section{Gráfico da Função Quadrática}
		\section{A Função Quadrática na Física}

	\chapter{Geometria Espacial}
		
		\section{Paralelelepípedos Retângulos}
			\subsection{Área Superficial}
			\subsection{Volume}
			\subsection{Diagonal}
		
		\section{Prismas e Pirâmides}
			\subsection{Área Superficial}
			\subsection{Volume}
		
		\section{Cilindros, Cones e Esferas}
			\subsection{Área Superficial}
			\subsection{Volume}
			
		\section{Volumes: Troncos de Cones e Troncos de Pirâmides}
		
		\section{Geometria Espacial de Posição}		

	\chapter{Análise Combinatória}
	
		
		\section{Introdução}
		
		\section{Permutações}
			\subsection{Permutações Simples}
			\subsection{Permutações Circulares}
			\subsection{Permutações com Repetição}
			\subsection{Problemas e Exercícios}
		
		\section{Combinações}
			\subsection{Combinações Simples}
			\subsection{Combinações Completa}
			\subsection{Problemas e Exercícios}
		
		\section{Problemas e Exercícios Gerais de Análise Combinatória}

\part*{III Unidade}
\chapter{Trigonometria}
	\section{Teorema de Tales}
		\subsection{Teorema de Tales}
		\subsection{Semelhança de Triângulos}
		\subsection{Relações Métricas num Triângulo Qualquer}
	\section{Triângulo Retângulo}
		\subsection{Teorema de Pitágoras}
		\subsection{Relações Métricas no Triângulo Retângulo}
		\subsection{Trigonometria no Triângulo Retângulo}
	\section{Trigonometria num Triângulo Qualquer}
		\subsection{Lei dos Senos}
		\subsection{Lei dos Cossenos}
\chapter{Trigonometria no Ciclo}
\chapter{Vetores no Plano}
\chapter{Funções Exponenciais}
\chapter{Porcentagens}
\chapter{Proporcionalidade e Regra de Trê}
\chapter{Probabilidade}

\part*{IV Unidade}

\chapter{Números Binomiais}
	\section{Triângulo de Pascal}
	\section{Binômio de Newton}

\chapter{Estatística}

\chapter{Propriedades dos Conjuntos Numéricos}

\chapter{Figuras Poligonais}

\chapter{Volumes}

%--------------------------------------------------------------------------------------------------------------------------------
%--------------------------------------------------------------------------------------------------------------------------------

\part{3ª Série do Ensino Médio}

\part*{I Unidade}
\chapter{Porcentagem}
	\section{Porcentagem}
	\section{Juros Simples e Composto}

\chapter{Polinômios}
\chapter{Números Complexos}
\chapter{Geometria Analítica}
	
\part*{II Unidade}
\section{Geometria Analítica: Retas e Circunferência}

	\subsection{Retas no Plano}
	\subsection{Circunferência no Plano}

\section{Função Seno}

	\subsection{Aplicações da Função seno na Física}	
	

\part*{III Unidade}

\chapter{Geometria Analítica: Cônicas}

\chapter{Função Cosseno}

\chapter{Volumes}

\chapter{Análise Combinatória}
	\section{Arranjos e Combinações}
	\section{Probabilidade}

\section{Vetores}

\chapter{•}
\part*{IV Unidade}

\chapter{Estatística}

\chapter{Volumes}

\chapter{Proporcionalidade}

	\section{Proporcionalidade Direta e Inversa}
	\section{Regra de Três Simples}
	\section{Regra de Três Composta}
	\section{Porcentagem}

\chapter{As Funções Trigonométricas e Movimento Circular}

\chapter{Operações com vetores}

%--------------------------------------------------------------------------------------------------------------------------------
%--------------------------------------------------------------------------------------------------------------------------------


\part{Anexos}
\chapter{Aulas Clube de Matemática}
\chapter{Proposta Curricular para o Ensino Médio - PE}
	\section{1 ª Série}
		\subsection{I Unidade}
		\subsection{II Unidade}
		\subsection{III Unidade}
		\subsection{IV Unidade}
	
	\section{2 ª Série}
		\subsection{I Unidade}
		\subsection{II Unidade}
		\subsection{III Unidade}
		\subsection{IV Unidade}
		
	\section{3 ª Série}
		\subsection{I Unidade}
		\subsection{II Unidade}
		\subsection{III Unidade}
		\subsection{IV Unidade}
	

\chapter{Resposta dos Problemas e Exercícios Propostos}

\end{document}
