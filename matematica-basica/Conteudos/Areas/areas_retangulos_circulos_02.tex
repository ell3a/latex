\documentclass[twocolumn,oneside,a4paper,10pt]{article}
\input{preambulo2.tex}

\begin{document}
\maketitle

\thispagestyle{empty}
\begin{enumerate}
\item Na figura a seguir temos uma coroa circular, formada por dois círculos concêntricos:
\figura{coroa_circular}{3cm}
\noindent Calcule a área da coroa sabendo que o círculo maior tem raio 20 cm e o círculo menor tem raio 15 cm:

\item Na coroa circular a seguir o círculo menor tem perímetro 200 cm e círculo maior tem perímetro 300 cm.
\figura{coroa_circular}{3cm}
\noindent Calcule a área da coroa circular:

\item Calcule a área hachura da figura a seguir, sabendo que os círculos menos tem raio 5 cm e o círculo maior tem raio 10:
\figura{circulo_circulos}{3cm}

\item O círculo maior da figura a seguir tem raio 28 cm, e os círculos menores tem raio 14 cm.
\figura{circulo_circulos}{3cm}

\noindent Calcule a área da área hachurada da figura anterior.

\item Um quadrado de lado 10 cm está inscrito em um círculo, conforme a figura a seguir:
\figura{circulo_quadrado}{3cm}

\noindent Calcule a área da área hachurada da figura anterior.

\item Calcule a área hachurada da figura anterior sabendo que o lado do quadrado é 25 cm:
\figura{circulo_quadrado}{3cm}

\item Na figura a seguir um círculo está circunscrito a um quadrado de área 221 cm\m{^2}, qual a área da parte hachurada da figura:
\figura{circulo_quadrado}{3cm}

\item Na figura a quadrado \m{ABCD} a seguir está inscrito em um círculo de raio 25 cm. Calcule a área da região hachurada da figura: 
\figura{circulo_quadrado_2}{3.5cm}

\newpage
\thispagestyle{empty}
\item Calcule a área hachurada da figura a seguir sabendo que o raio do círculo é 80 cm.
\figura{circulo_quadrado_2}{3.5cm}

\item O círculo da figura a seguir, inscrito em um quadrado, tem raio 60 cm, calcule a área da parte hachurada:
\figura{quadrado_circulo}{3cm}

\item Sabendo que a área do círculo é 800 cm\m{^2}, calcule a área da parte hachurada da figura a seguir:
\figura{quadrado_circulo}{3cm}

\item O círculo a seguir tem área 90 cm\m{^2}, calcule a área da parte hachurada da figura:
\figura{quadrado_circulo}{3cm}

\item Na figura a seguir, temos um círculo inscrito de em quadrado.
\figura{quadrado_circulo}{3cm}
\noindent Sabendo que a área do quadrado é 400 cm\m{^2}, calcule:
\begin{enumerate}
\item Qual o lado quadrado?
\item Qual o raio círculo?
\item Qual o perímetro do círculo?
\item Qual a área do círculo?
\item Qual a área da parte hachurada da figura?
\end{enumerate}

\item Na figura a seguir, temos um círculo inscrito de em quadrado.
\figura{quadrado_circulo}{3cm}
\noindent Sabendo que a área do cículo é 1600 cm\m{^2}, calcule:
\begin{enumerate}
\item Qual o raio círculo?
\item Qual o perímetro do círculo?
\item Qual a medida do lado quadrado?
\item Qual a área quadrado?
\item Qual a área da parte hachurada da figura?
\end{enumerate}

\item O retângulo a seguir, inscrito em um círculo, tem lados \m{AB = 30 cm} e \m{BC = 40 cm}, calcule a área da parte hachurada da figura:
\figura{circulo_retangulo}{4cm}

\item Na figura a seguir temos um retângulo de lados  \m{AB = 50 cm} e \m{BC = 120 cm}. Qual a o valor da área da figura hachurada:
\figura{retangulo_semi_circulo}{5cm}

\item Calcule a área da figura a seguir sabendo que \m{AD = 100 cm} e \m{AB = 350 cm}
\figura{semi_circulos_retangulo}{7cm}
\end{enumerate}


\begin{flushright}
\textit{\textbf{Pedaços}\\
Tudo que amamos são pedaços vivos do nosso próprio ser.}
\end{flushright}
\end{document}