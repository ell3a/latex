\usepackage[english,brazilian]{babel}
\usepackage[alf]{abntex2cite}
\usepackage[utf8]{inputenc}
\usepackage[T1]{fontenc}
\usepackage[top=10mm, bottom=15mm, left=10mm, right=10mm]{geometry}
\usepackage{framed}
\usepackage{booktabs}
\usepackage{color}
\usepackage{hyperref}
\usepackage{graphicx,subfigure}
\usepackage{float}
\graphicspath{{./Figuras/}}    
\definecolor{shadecolor}{rgb}{0.8,0.8,0.8}

\usepackage{amsthm}
\usepackage{amsmath}
\usepackage{amsfonts}

\usepackage[none]{hyphenat} %Evita a hifenização no documento

\usepackage{multicol}
\columnsep=10mm %Espaçamento entre colunas.
\setlength{\columnseprule}{1.25pt}

\newcommand{\figura}[2]{\begin{figure}[!htb]\center\includegraphics[width=#2]{#1}\end{figure}}

\newcommand{\LINHAHORIZONTAL}{\center \rule{16cm}{1.25pt}}
\title{\LINHAHORIZONTAL \\ \textbf{Atividade de Matemática: Áreas I}}
\author{Prof. Leandro Vieira\\\textbf{EREM Regina Pacis}\\3 EMSI A}
\date{outubro de 2020\\ \LINHAHORIZONTAL}

\newcommand{\m}[1]{\(\displaystyle{#1}\)}
\newcommand{\M}[1]{\[{#1}\]}
\newtheorem{exemplo}{exemplo}
