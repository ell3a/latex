\documentclass[twocolumn,oneside,a4paper,11pt]{article}
\usepackage[english,brazilian]{babel}
\usepackage[alf]{abntex2cite}
\usepackage[utf8]{inputenc}
\usepackage[T1]{fontenc}
\usepackage[top=10mm, bottom=15mm, left=10mm, right=10mm]{geometry}
\usepackage{framed}
\usepackage{booktabs}
\usepackage{color}
\usepackage{hyperref}
\usepackage{graphicx,subfigure}
\usepackage{float}
\graphicspath{{./Figuras/}}    
\definecolor{shadecolor}{rgb}{0.8,0.8,0.8}

\usepackage{amsthm}
\usepackage{amsmath}
\usepackage{amsfonts}

\usepackage[none]{hyphenat} %Evita a hifenização no documento

\usepackage{multicol}
\columnsep=10mm %Espaçamento entre colunas.
\setlength{\columnseprule}{1.25pt}

\newcommand{\figura}[2]{\begin{figure}[!htb]\center\includegraphics[width=#2]{#1}\end{figure}}

\newcommand{\LINHAHORIZONTAL}{\center \rule{16cm}{1.25pt}}
\title{\LINHAHORIZONTAL \\ \textbf{Atividade de Matemática: Áreas I}}
\author{Prof. Leandro Vieira\\\textbf{EREM Regina Pacis}\\3 EMSI A}
\date{outubro de 2020\\ \LINHAHORIZONTAL}

\newcommand{\m}[1]{\(\displaystyle{#1}\)}
\newcommand{\M}[1]{\[{#1}\]}
\newtheorem{exemplo}{exemplo}


\begin{document}
\maketitle

\section*{Áreas e Perímetros}
\subsection*{Retângulo}
Um retângulo é uma figura plana de formada por quarto lados e quatro ângulos retos.

\figura{r0}{3cm}

\noindent No retângulo anterior, temos \m{AB = CD} e \m{BC = AD}. O perímetro de um retângulo é a soma da medida de seus lados, e a área é o produto da medida de dos lados adjacentes, assim para o retângulo da figura anterior temos:
\begin{itemize}
\item perímetro \m{= AB+BC+CD+AD};
\item área \m{=  AB \times BC} ou \m{A=AD \times CD}.
\end{itemize}

\begin{exemplo}
 
\end{exemplo}

\begin{exemplo}
A frente de um terreno retangular mede 20 m. Sabendo que para cercá-lo completamente são necessários 180 m de cerca, calcule a área do terreno:
\end{exemplo}

\begin{exemplo}
Um retângulo de área 180 cm\m{^2}, tem um de seus medindo 15 cm. Calcule o perímetro desse retângulo:
\end{exemplo}

\begin{exemplo}
A medida dos fundos de um terreno retangular é 16 m, sabendo a que área desse retângulo é 400 m\m{^2}, calcule o seu perímetro:
\end{exemplo}

\begin{exemplo}
Um quadrado tem área 289 cm\m{^2}. Calcule seu perímetro:
\end{exemplo}

\begin{exemplo}
Qual o área de um quadrado que tem 180 m de perímetro:
\end{exemplo}

\

\subsection*{Círculos}

Dado um ponto \m{C} no plano e um número real positivo \m{r}, a circunferência de centro \m{C} e raio \m{r}, é conjunto de pontos do plano distantes \m{r} de \m{C}: 
\figura{c0}{4cm}

\noindent Na figura anterior temos um circunferência de centro \m{C} e raio \m{r}. Note que Os pontos \m{P} e \m{Q} na circunferência estão a uma mesma distância \m{r} do centro. O círculo é formado por toda a região interna ao círculo. Para o círculo temos as seguintes fórmulas:
\begin{itemize}
\item perímetro = circunferência: \m{C = 2 \pi r};
\item área: \m{A = \pi r^2}.
\end{itemize}

\begin{exemplo}
Qual a área de um círculo que tem 250 cm de perímetro:
\end{exemplo}

\begin{exemplo}
Um terreno circular tem uma circunferência de 2580 m. Calcule a área desse terreno:
\end{exemplo}

\begin{exemplo}
Calcule o perímetro de um círculo de área 980 cm\m{^2}:
\end{exemplo}

\begin{exemplo}
Qual a circunferência de um terreno circular cuja medida de sua área é 750 m\m{^2}:
\end{exemplo}

\newpage
\section*{Exercícios}
\begin{enumerate}
\item O fundos e as laterais do terreno retangular a seguir serão cercados.
\figura{terreno01}{5cm}

Sabendo que o comprimento da cerca é 34 m, e que o os fundos medem 10 m, calcule a área do terreno:

\item O fundos e as laterais de um terreno retangular serão cercados. Sabendo que o comprimento da cerca é 50 m, e que o os fundos medem 18 m, calcule a área do terreno:

\item Para cercar todos os lados de um terreno retangular de fundos 28 m não necessários 180 m de cerca. Qual a área do terreno?

\item Um terreno retangular tem uma área de 500m\m{^2}, sabendo que a frente do terreno mede 20 m, calcule o seu perímetro:

\item Quantos metros de cerca serão necessários para cercar um terreno retangular cuja frente mede 18 m, e sua área é 648 \m{^2}:

\item Um arquiteto projetou uma praça circular conforme o modelo a seguir
\figura{praca_circular_01}{4.5cm}

Sabendo a circunferência da praça é 200 m, calcule sua área.

\item Uma praça circular tem perímetro 500 m. Calcula qual a área da praça:

\item Calcule a área da praça circular a seguir sabendo que seu perímetro é 150 m:
\figura{praca_circular_02}{4.5cm}

\item Qual a área de um círculo de cuja circunferência mede 180 cm?



\item Um arquiteto tem um projeto de construção de uma praça circular conforme a figura a seguir
\figura{praca_circular_03}{4cm}

Sabendo que a área do a praça deve ser 2500~m\m{^2}, calcule seu perímetro:

\item Um terreno circular como o representado a seguir tem área igual a 150 m\m{^2}:

\figura{terreno_circular_01}{4cm}

Calcule seu perímetro.

\item Um terreno circular 1200 m\m{^2} de área. Calcule qual o perímetro do terreno?

\item Qual a circunferência de um círculo que tem 2800 cm\m{^2} de área:

\item A praça circular da figura a seguir deverá ter uma área de 1590,43 m\m{^2}:
\figura{praca_circular_04}{5cm}

\item Calcule a área de um círculo que tem perímetro de medida 2000 km:

\item Calcule o perímetro de um círculo que tem área 2000 km\m{^2}:

\item Qual área de um quadrado cujo perímetro mede 160 cm:

\item Qual o perímetro de um quadrado cuja área mede 2500 cm\m{^2}:

\end{enumerate}

\begin{flushright}
\textit{\textbf{Tempo e Amizade}\\
Há amigos de oito dias e indiferentes de oito anos.}
\end{flushright}
\end{document}