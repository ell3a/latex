\documentclass[twocolumn,oneside,a4paper,12pt]{article}
\input preambulo.tex

%\begin{figure}[!htb]
%\center
%\includegraphics[width=15cm]{capa}
%\end{figure}

\begin{document}

\maketitle

\section*{Sistemas \m{2 \times 2}}

Seja o sistema a seguir:

\begin{equation*}
  \left\{ \begin{aligned}
      2x - 5y & = -7 \\
      7x + 4y & = 8 
  \end{aligned} \right.
\end{equation*}

\noindent Podemos transformá-lo na seguinte equação matricial:

\M{A \times X = C,}

\noindent onde 

\M{
A = \begin{bmatrix}
   2 & -5  \\
   7 & 4 
  \end{bmatrix},
\mbox{ }  
X = \begin{bmatrix}
   x   \\
   y 
  \end{bmatrix} 
\mbox{ e }
C = \begin{bmatrix}
   -7   \\
   8 
  \end{bmatrix}
}

Multiplicando a equação matricial \m{A \times X = C} à esquerda por \m{A^{-1}}, obtemos:
\begin{eqnarray*}
A^{-1} \times (A \times X) & = & A^{-1} \times C \\
(A^{-1} \times A) \times X & = & A^{-1} \times C \\
I \times X & = & A^{-1} \times C \\
X & = & A^{-1} \times C
\end{eqnarray*}

\noindent Dessa forma a resolução do sistema passar a ser uma questão de multiplicação de matrizes, sabendo a inversa da matrix \m{A}:

\M{
A^{-1} = \begin{bmatrix}
   \frac{4}{43} & \frac{5}{43}  \\
   -\frac{7}{43} & \frac{2}{43} 
  \end{bmatrix}
}

\noindent E portanto

\M{
X = \begin{bmatrix}
   \frac{4}{43} & \frac{5}{43}  \\
   -\frac{7}{43} & \frac{2}{43} 
  \end{bmatrix} \times \begin{bmatrix}
   -7   \\
   8 
  \end{bmatrix} = \begin{bmatrix}
   \frac{12}{43}   \\
   \frac{65}{43} 
  \end{bmatrix}
}

Outro exemplo de sistema que resolveremos como a anterior vem a seguir:

\begin{equation*}
  \left\{ \begin{aligned}
      3x + 4y - 5z & = 37 \\
      8x - 6y + 9z & = -12 \\
      -11x + 146y + 2z & = -7
  \end{aligned} \right.
\end{equation*}

\noindent que pode ser analisado como a seguinte equação matricial:

\M{A \times X = C}
\M{
\begin{bmatrix}
   3 & 4 & -5 \\
   8 & -6 & 9 \\
   -11 & 14 & 2
\end{bmatrix} \times \begin{bmatrix}
   x \\
   y \\
   z 
  \end{bmatrix} = \begin{bmatrix}
   37 \\
   -12 \\
   -7
  \end{bmatrix}
}



\end{document}