\documentclass[12pt,a4paper]{article}

\usepackage[T1]{fontenc} % dofificação da fonte em 8-bits
\usepackage[utf8]{inputenc} % acentuação direta
\usepackage[brazil]{babel} % em portugues brasileiro

% regra de hifenização das palavras nao acentuadas:
% nao requer \usepackage[T1]{fontenc}
\hyphenation{li-vro tes-te cha-ve bi-blio-te-ca}
% regra de hifenização das palavras acentuadas:
% requer \usepackage[T1]{fontenc}
\hyphenation{comentário}

\usepackage{amsfonts}
\usepackage{amssymb} 
%-----------------------------------------------------------------
\usepackage{amscd}
% diagrama comutativa
%-----------------------------------------------------------------
% eucal troca a fonte mathcal pela fonte caligrafica
% Euler Script que eh mais enfeitada que do
% computer Modern
\usepackage{eucal}
% a opcao mathscr mantem a versao original do mathcal e 
% cria o comando mathscr para Euler Script.
% \usepackage[mathscr]{eucal}
% nota: \usepackage{eucal} e mesmo que 
%   \usepackage[eucal]{euscript}
%   e \usepackage[mathscr]{euscript} e mesmo que
%   \usepackage[mathscr]{euscript}
%----------------------------------------------------------------
% amsmath carrega diversos pacotes do AMS para incrementar
% o ambiente matematico
% Os pacotes amssymb e amsmath sao mais usados
\usepackage{amsmath}


%%%%%%%%%%%%%%%%%%%%%%%%%%%%%%%%%%%%%%%%%%%%
\newcommand{\Rset}{\mathbb{R}}
\newcommand{\Cset}{\mathbb{C}}
\newcommand{\Qset}{\mathbb{Q}}
\newcommand{\Iset}{\mathbb{I}}
\newcommand{\Zset}{\mathbb{Z}}
\newcommand{\Nset}{\mathbb{N}}

%%%%%%%%%%%%%%%%%%%%%%%%%%%%%%%%%%%%%%%%%%
% amsmath oferece \DeclaremathOperator para 
% definir comandos associados as funcoes matematicas
% \newcommand{\sen}{\mathrm{sen}}
\DeclareMathOperator{\sen}{sen} % mesmo efeito
% do \newcommand acima
% A versão com * define como estilo
% \sum, \lim, \sup, \inf, etc, isto eh,
% no modo displaystyle, poderá coocar limitantes
% emcima e embaixo
\DeclareMathOperator*{\supinf}{sup\,inf} % limitante como
% no \sum

%%%%%%%%%%%%%%%%%%%%%%%%%%%%%%%%%%%%%%%%%%%%%%%%%%%%%%%%
\begin{document} % inicio do documento

Um matrix pode ser criado com o uso do ambiente \verb+matrix+. 
O separador da coluna é

\& e o separador da linha é \verb|\\| (newline).

\[
  \begin{matrix}
   1 & 2 & 3 \\
   2 & 3 & 4 \\
   3 & 4 & 5  
  \end{matrix}
\]

Para matrizes com delimitadores, existem os comandos que colocam automaticamente
os delimitadores ajustáveis: 
\verb+pmatrix+ (parenteses), \verb+bmatrix+ (colchetes), 
\verb+Bmatrix+ (chaves), \verb+vmatrix+ (linha vertical), \verb+Vmatrix+ (linha vertical dupla):

\[
  \begin{matrix}
   1 & 2 & 3 \\
   2 & 3 & 4 \\
   3 & 4 & 5  
  \end{matrix},
  \begin{pmatrix}
   1 & 2 & 3 \\
   2 & 3 & 4 \\
   3 & 4 & 5  
  \end{pmatrix}, 
  \begin{bmatrix}
   1 & 2 & 3 \\
   2 & 3 & 4 \\
   3 & 4 & 5  
  \end{bmatrix}, 
  \begin{Bmatrix}
   1 & 2 & 3 \\
   2 & 3 & 4 \\
   3 & 4 & 5  
  \end{Bmatrix}, 
  \begin{vmatrix}
   1 & 2 & 3 \\
   2 & 3 & 4 \\
   3 & 4 & 5  
  \end{vmatrix},
  \begin{Vmatrix}
   1 & 2 & 3 \\
   2 & 3 & 4 \\
   3 & 4 & 5  
  \end{Vmatrix}
\]

Para quebrar uma equação em mais de uma linha, podemos usar o ambiente \verb+split+.

\[
\begin{split}
2x+y=3 \\
x-y=1+a
\end{split}
\]

Para criar alinhamento na equação quebrada, usa-se 
\verb|aligned| em vez de \verb|split| e insere o \& nos pontos de alinhamento.

\begin{equation}
\begin{aligned}
2x+y&=3 \\
x-y&=1+a
\end{aligned}
\end{equation}


Para definir a função por partes, é útil usar o ambiente \verb+cases+:

\[
f(x) = 
  \begin{cases}
      x^2, & x<0 \\
      x, & x \geq 0 
  \end{cases}
\]

Podemos tentar usar tambm para definir sistema de equaes como
\begin{equation}
  \begin{cases}
      2x+y=1 \\
      x-y=1+a 
  \end{cases}
\end{equation}
mas isto não efetua o alinhamento (na posição de igualdade). 
Não podemos usar o  \& pois o alinhamento seria efetuado de forma 
prÃprio para colocar condição da expressão e não a equação.
Assim, é aconselhável usar o \verb+aligned+ em vez de \texttt{cases}.

\begin{equation}
  \left\{ \begin{aligned}
      2x+y&=1 \\
      x-y&=1+a 
  \end{aligned} \right.
\end{equation}

Existem diversos ambientes estilo \texttt{equations}. 
Vamos ver algumas delas:

O ambiente \texttt{gather} produz ves, todas centradas:

\begin{gather}
x+y+z=1 \\
x-y+z=2 \\
x+y=0
\end{gather}

Para inibir enumeração em algumas delas, coloque o \verb|\nonumber| na equação
ue deseja remover a enumeração (antes da quebra de linhas).

\begin{gather}
x+y+z=1 \\
x-y+z=2 \nonumber \\
x+y=0
\end{gather}

Para equas em vÃrias linhas, com pontos de alinhamento, 
usa-se o \texttt{align} que pode inibir enumeraà 
de algumas equaes como em \texttt{gather}

\begin{align}
x+y+z&=1 \\
x-y+z&=2 \nonumber \\
x+y&=0
\end{align}

Quando escreve uma expressão grande quebrado em várias linhas,
as vezes é legal tabular cada linha para direita.
Este efeito pode ser obtido pelo \verb|multline|.

\begin{multline}
Ax+Ay+Az=
  a_{11}x_1+\cdots +a_{1n}x_n+\cdots +\cdots +a_{n1}x_1+\cdots a_{nn}x_n    \\
  a_{11}y_1+\cdots +a_{1n}y_n+\cdots +\cdots +a_{n1}y_1+\cdots a_{nn}y_n    \\
  a_{11}z_1+\cdots +a_{1n}z_n+\cdots +\cdots +a_{n1}z_1+\cdots a_{nn}z_n    
\end{multline}


Para mais ambientes deste estilo, consulte o manual de AMS \LaTeX.

Note que, todos ambientes do estilo \texttt{equation} 
(\texttt{equation}, \texttt{gather}, \texttt{align}, \texttt{multline}, etc)
apresenta a versão com * na qual remove toda enumeração (versão * equivale 
a colocar \verb|\nonumber| em todas).

\begin{align*}
x+y+z&=1 \\
x-y+z&=2 \\
x+y&=0
\end{align*}

Como exercício, tente referÃnciar as equaÃs enumeradas produzidas pelos ambientes acima,
(usando \verb|\label{}| e \verb|\ref{}|).

\

\textbf{Observação:} O \texttt{eqnarray} oferecido pelo \LaTeX{} inclui 
espaço extra nos pontos de alinhamento e portanto, 
deve ser evitados.
Observe o espaço extra antes e depois de $=$ no \texttt{eqnarray} que segue.

\begin{eqnarray*}
x+y+z&=&1 \\
x-y+z&=&2 \\
x+y&=&0
\end{eqnarray*}


\end{document} % final do documento
