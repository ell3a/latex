\documentclass[oneside,a4paper,12pt]{article}
\usepackage[english,brazilian]{babel}
\usepackage[alf]{abntex2cite}
\usepackage[utf8]{inputenc}
\usepackage[T1]{fontenc}
\usepackage[top=25mm, bottom=20mm, left=20mm, right=20mm]{geometry}
\usepackage{framed}
\usepackage{booktabs}
\usepackage{color}
\usepackage{hyperref}
\usepackage{graphicx}
\usepackage{amsfonts}
\usepackage{subfigure}
\usepackage{enumerate}
\usepackage{float}
\graphicspath{{./Figuras/}}    
\definecolor{shadecolor}{rgb}{0.8,0.8,0.8}
\pagenumbering{arabic}

\usepackage{pgf,tikz}
\usetikzlibrary{arrows}

\usepackage{multicol}
\setlength{\columnseprule}{0pt}

%Cabeçario
\usepackage{fancyhdr}
\lhead{\rightmark}
\renewcommand{\footrulewidth}{0.5pt}

\setlength{\parindent}{1.25cm}%paragrafo

\pagestyle{fancy}
\def\MakeUppercase{} %Fonte minúscula no cabeçario

\pagenumbering{arabic}

\usepackage{amsthm}
\newtheorem{exem}{Exemplo}[section]

\usepackage[skip=10pt]{caption}
\captionsetup{font={stretch=0.4,small}} 

\newcommand{\m}[1]{\({#1}\)}
\newcommand{\M}[1]{\[{#1}\]}
\newcommand{\sol}{\noindent \textbf{Solução }}

\begin{document}
\begin{enumerate}
\item Myanmar é um país do sul da Ásia continental que faz divisa, dentre outros países, com a China e  a Índia. Nos gráficos a seguir são apresentadas algumas informações sobre características da população 

\begin{figure}[!hbt]
\begin{multicols}{2}
\includegraphics[width=4.5cm]{01}
\includegraphics[width=4.5cm]{02}

De acordo com dados de 2014 sua população seria de aproximada 51.500.000. Com base nas informações apresentadas, resolva aos seguinte itens:
\begin{enumerate}
\item 

\item 
\end{enumerate}

\end{multicols}
\end{figure}

\item O Butão é um país localizado no sul da Ásia, no extremo leste dos Himalaias. Faz fronteira ao norte com a China e ao sul, leste e oeste com a índia. A seguir é apresentada uma tabela com informações sobre as cidades mais populosas do Butão:

\begin{figure}[!hbt]
\begin{multicols}{2}
\includegraphics[width=8cm]{03}

\columnbreak
Com base nas informações apresentadas na tabela, sabendo que a população do Butão é de 831 557 , resolva os itens a seguir:
\begin{enumerate}
\item 

\item 
\end{enumerate}

\end{multicols}
\end{figure}

\newpage
\item O gráfico a seguir apresenta dados sobre faturamento de uma pequena lanchonete em cinco meses do ano de 2020.

\begin{figure}[!hbt]
\begin{multicols}{2}
\includegraphics[width=10cm]{04}

\columnbreak
A coluna é o investimento em material e estrutura feito pelo dono da lanchonete, o gráfico em preto com linha contínua é o gráfico de lucro para os meses em questão, já o gráfico tracejado representa o faturamento na lanchonete no período observado:

Com base nas informações apresentadas na no gráfico resolva aos seguintes itens:
\begin{enumerate}
\item Em qual dos meses o lucro foi o maior? Qual o valor desse lucro? 
\item Em qual mês o faturamento foi o menor? Qual o valor do faturamento?
\item Existem meses que o faturamento foi o mesmo? Se sim, quais?
\item Existem meses que o lucro foi o mesmo? Se sim, quais?
\end{enumerate}

\end{multicols}
\end{figure}


\end{enumerate}
\end{document}