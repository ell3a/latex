\usepackage[english,brazilian]{babel}
\usepackage[alf]{abntex2cite}
\usepackage[utf8]{inputenc}
\usepackage[T1]{fontenc}
\usepackage[top=10mm, bottom=10mm, left=10mm, right=10mm]{geometry}
\usepackage{framed}
\usepackage{booktabs}
\usepackage{color}
\usepackage{hyperref}
\usepackage{graphicx,subfigure}
\usepackage{float}
\graphicspath{{./Figuras/}}    
\definecolor{shadecolor}{rgb}{0.8,0.8,0.8}

\usepackage[none]{hyphenat} %Evita a hifenização no documento

\newcommand{\ESCOLA}{EREM Regina Pacis}
\newcommand{\SERIE}{3 EMSI A}
\newcommand{\PROFESSOR}{\textbf{Prof. Leandro Vieira}}
\newcommand{\DATA}{\noindent \textbf{junho de 2020: análise de gráficos}}
\newcommand{\CONTEUDO}{{\LARGE \textbf{Atividade de Matemática}}}
\newcommand{\FRASE}{\flushright{"A maior lição da vida é a de que, às vezes, até os tolos têm razão."\\\textbf{Winston Churchill}}}

\newcommand{\cabecalho}{\DATA \\ \begin{center}\ESCOLA \\ \SERIE \\ \PROFESSOR \\ \vspace{10pt} \CONTEUDO \end{center}}

\usepackage{amsthm}
\newtheorem{exemplo}{Exemplo}

\usepackage{enumerate}
\usepackage{multicol}
\columnsep=10mm %Espaçamento entre colunas.

\title{\textbf{Permutações}}
\author{Leandro Vieira\\EREM Regina Pacis}