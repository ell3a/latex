\documentclass[twocolumn,oneside,a4paper,12pt]{article}
\usepackage[english,brazilian]{babel}
\usepackage[alf]{abntex2cite}
\usepackage[utf8]{inputenc}
\usepackage[T1]{fontenc}
\usepackage[top=25mm, bottom=20mm, left=20mm, right=20mm]{geometry}
\usepackage{framed}
\usepackage{booktabs}
\usepackage{color}
\usepackage{hyperref}
\usepackage{graphicx}
\usepackage{amsfonts}
\usepackage{subfigure}
\usepackage{enumerate}
\usepackage{float}
\graphicspath{{./Figuras/}}    
\definecolor{shadecolor}{rgb}{0.8,0.8,0.8}
\pagenumbering{arabic}

\usepackage{pgf,tikz}
\usetikzlibrary{arrows}

\usepackage{multicol}
\setlength{\columnseprule}{0pt}

%Cabeçario
\usepackage{fancyhdr}
\lhead{\rightmark}
\renewcommand{\footrulewidth}{0.5pt}

\setlength{\parindent}{1.25cm}%paragrafo

\pagestyle{fancy}
\def\MakeUppercase{} %Fonte minúscula no cabeçario

\pagenumbering{arabic}

\usepackage{amsthm}
\newtheorem{exem}{Exemplo}[section]

\usepackage[skip=10pt]{caption}
\captionsetup{font={stretch=0.4,small}} 

\newcommand{\m}[1]{\({#1}\)}
\newcommand{\M}[1]{\[{#1}\]}
\newcommand{\sol}{\noindent \textbf{Solução }}

\begin{document}

\begin{enumerate}

\item As informações a seguir são de estudante da terceira série do ensino médio de uma escola.

\begin{tabular}{cccc}
\hline 
 & Zona Urbana & Zona Rural & Total \\ 
\hline 
Homens & 12 & 8 &  \\ 
Mulheres & 9 & 10 &  \\ 
\hline
\end{tabular} 

Serão sorteados alunos dentre os discentes dessa turma. Com base nas informações mostradas, responda aos seguintes itens:
\begin{enumerate}[(a)]
\item Qual a probabilidade de sortear uma mulher?
\item Qual a probabilidade de sortear um homem, sabendo que ele é da zona rural?
\item Qual a probabilidade de sortear um estudante residente na zona rural na certeza que é uma mulher?
\item Qual a probabilidade de sortear um homem, dado que é da zona urbana?
\end{enumerate}

\end{enumerate}

\begin{flushright}
\textit{Viver é a coisa mais rara do mundo.\\A maioria das pessoas apenas existe.\\\textbf{Oscar Wilde}}
\end{flushright}
\end{document}