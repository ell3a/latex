\documentclass[twocolumn,oneside,a4paper,12pt]{article}
% ---------------- Para Modificar ---------------- 
\newcommand{\principal}{Volumes}
\newcommand{\conteudo}{}
\newcommand{\turmas}{3~EMSI~A e do 3~EMSI~B}

\date{abril de 2021}

\newcommand{\citacao}{Que nada nos defina. Que nada nos sujeite. Que a liberdade seja a nossa própria substância.}
\newcommand{\autorcitacao}{Simone de Beauvoir}
% ------------------------------------------------

%-------------------------------------------------
\usepackage[english,brazilian]{babel}
\usepackage[alf]{abntex2cite}
\usepackage[utf8]{inputenc}
\usepackage[T1]{fontenc}
\usepackage[top=15mm, bottom=15mm, left=10mm, right=10mm]{geometry}
\usepackage{framed,booktabs,color,hyperref,graphicx}
\usepackage{amsfonts,amsthm,cancel}
\usepackage{subfigure,enumerate,float}
  
\definecolor{shadecolor}{rgb}{0.8,0.8,0.8}
\pagenumbering{arabic}

% Colunas
\usepackage{multicol}
\columnsep=10mm %Espaçamento entre colunas.
\setlength{\columnseprule}{1pt}

% Cabeçalho
\usepackage{fancyhdr}
\pagestyle{fancy}
\lhead{\textbf{\principal}}
\rhead{}
\renewcommand{\headrulewidth}{1pt} % espessura da linha do cabeçalho
\renewcommand{\footrulewidth}{1pt} % espessura da linha do rodapé

% Parágrafo
\setlength{\parindent}{1.25cm}

\newtheorem{problema}{Problema}
\newtheorem{exercicio}{exercicio}
\newtheorem{exemplo}{Exemplo}
\newtheorem{questao}{Questão}

\usepackage[skip=10pt]{caption}
\captionsetup{font={stretch=0.4,small}}

\newcommand{\FRASE}{\textit{``\citacao ''}\\(\textbf{\autorcitacao})}

\title{\LINHAHORIZONTAL \\\textbf{\\ \principal}\footnote{Resumo para os estudos das aulas não presenciais no período de quarentena para as turmas do \turmas .}\\\LINHAHORIZONTAL}

\newcommand{\LINHAHORIZONTAL}{\center \rule{16cm}{1.25pt}}
\newcommand{\sol}{\textbf{Solução}}

\newcommand{\m}[1]{\(\displaystyle {#1}\)}
\newcommand{\M}[1]{\[{#1}\]}

\author{\textbf{Professor Leandro Vieira}\\EREM Regina Pacis\\Palmeirina-PE}
\newcommand{\frase}{\begin{verse} \flushright{\FRASE} \end{verse}}


\begin{document}
\pagestyle{empty}
\begin{center}
\textbf{Atividade de Matemática\\Porcentagem}
\end{center}

\begin{enumerate}

\item Calcule:
\begin{enumerate}
\item 30\% de 1500.
\item 12\% de 120. 
\item 27\% de 900.
\item 55\% de 300.
\item 98\% de 450
\end{enumerate}

\item Sabendo que 45\% de um número equivalem a 36, determine esse número.

\item Em uma turma de 40 alunos, 45\% são meninos. Quantos  meninos  e  meninas  tem a turma:

\item João comprou uma TV e resolveu pagar à prazo, pois não podia pagar à vista. Sabendo que o valor à vista é de R\$ 1500,00 e que o valor total à prazo é 15\% maior que o valor à vista, responda: Quanto João vai pagar no total?

\item Uma  televisão  que custava  R\$  900,00  teve um   aumento   de   R\$   50,00.   Qual   foi   o percentual de aumento?

\item Um artigo esportivo teve um aumento de 20\%, e agora custa R\$ 180,00. Qual era o preço antes desse aumento?

\item Um trabalho de matemática tem 30 questões de aritmética e 50 de geometria. Júlia acertou 70\% das questões de aritmética e 80\% do total de questões. Qual o percentual das questões de geometria que ela acertou?

\item Um  terreno  que  custava  R\$  50.000,00  há dois anos teve  uma  valorização  de 16,5\% nos  últimos 24 meses. Qual o valor atual do terreno?

\item No intuito de reduzir o consumo de energia elétrica mensal das residências de um determinado país, o governo baixou uma medida provisória decretando que todos reduzam o consumo de energia em até 15\%. Essa medida foi criada para que não haja riscos de ocorrerem apagões, em razão da escassez de chuvas que deixaram os reservatórios das hidrelétricas abaixo do nível de segurança. Salvo que a água é utilizada na movimentação das turbinas geradoras de energia elétrica. De acordo com a medida provisória, uma residência com consumo médio de 652 quilowatts-hora mensais, terá que reduzir o consumo em quantos quilowatts-hora mensal?

\item Em uma escola há 800 alunos matriculados, dos quais 60\% praticam esportes. Desses 60\% temos que: 70\% praticam futebol, 20\% praticam vôlei e 10\% fazem natação. Determine o número de alunos que praticam futebol, vôlei e natação.

\item Numa comunidade com 320 pessoas sabe-se que 25\% são idosos e 40\% são crianças. Nessas condições o total de idosos e crianças dessa comunidade é: 

\item Um funcionário de uma empresa recebeu a quantia de R\$ 315,00 a mais no seu salário, referente a um aumento de 12,5\%. Sendo assim, o seu salário atual é de:

\item Três laboratórios produzem certo medicamento. A tabela abaixo mostra, para um certo mês, o número de unidades produzidas desse medicamento e a porcentagem de vendas dessa produção.
\figura{01}{6cm}
Se, nesse mês, os três laboratórios venderam um total de 13.900 unidades desse medicamento, então o valor de \m{x} é:

\item Segundo o censo do IBGE, em 2010, o Brasil tinha   147,4  milhões   de   pessoas   com   10 anos ou mais que eram alfabetizadas, o que correspondia  a  91\%  da  população  nessa faixa    etária.    Determine    o    número    de brasileiros com 10 anos ou mais em 2010.

\item Levantamento efetuado pela Secretaria de Educação de certo município mostrou que atos de violência física ou psicológica, intencionais e repetitivos (bullying), estiveram envolvidos em cinco de cada oito desavenças entre alunos ocorridas em determinado período.

Com base nessas informações, é correto afirmar que as desavenças não motivadas por bullying representam, do número total de desavenças ocorridas nesse período:

\item Numa padaria, em 60 kg de farinha e fermento, 2\% é de fermento. Se acrescentarmos mais 100 gramas de fermento, qual a porcentagem da nova mistura?

\item Pedro acertou 21 questões de uma prova, que correspondem a 70\% do total de questões. Quantas questões tinha a prova?

\item Em uma promoção, o preço de um objeto foi reduzido de R\$ 76,00 para R\$ 57,00. Calcule o valor do desconto em porcentagem.

\item Uma conta de restaurante, incluindo os 10\% de serviço, ficou em R\$ 143,00. Qual o valor da conta sem a taxa de serviço?

\item Se 30\% da população de uma cidade litorânea mora na área insular e os demais 337.799 habitantes moram na área continental. Quantas pessoas moram na ilha? 
 
\item De acordo com a Receita Federal, para cada faixa salarial acima de R\$ 1787,77 mensal, paga-se uma porcentagem referente ao imposto de renda. Confira a Tabela Progressiva para o cálculo mensal do imposto sobre a renda da pessoa física a partir do exercício de 2015, ano-calendário de 2014
\figura{02}{9cm}
Sabendo que Márcia ganha salário de R\$ 2500,00 por mês, calcule quanto ela deverá pagar de imposto, tendo em vista que a alíquota é calculada sobre a diferença entre o salário e a faixa de isenção (R\$ 1787,77).

\item O contribuinte que vende mais de R\$ 20 mil de ações em Bolsa de Valores em um mês deverá pagar Imposto de Renda. O pagamento para a Receita Federal consistirá em 15\% do lucro obtido com a venda das ações. Um contribuinte que vende por R\$ 34 mil um lote de ações que custou R\$ 26 mil terá de pagar de Imposto de Renda à Receita Federal o valor de:

\item Uma enquete, realizada em março de 2010, perguntava aos internautas se eles acreditavam que as atividades humanas provocam o aquecimento global. Eram três as alternativas possíveis e 279 internautas responderam à enquete, como mostra o gráfico:
\figura{03}{7cm}
Analisando os dados do gráfico, quantos internautas responderam ``NÃO'' à enquete:
\end{enumerate}

\begin{flushright}
\textit{Viver é a coisa mais rara do mundo.\\A maioria das pessoas apenas existe.\\\textbf{Oscar Wilde}}
\end{flushright}
\end{document}