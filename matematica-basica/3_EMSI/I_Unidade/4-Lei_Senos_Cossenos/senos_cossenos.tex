\documentclass[twocolumn,oneside,a4paper,12pt]{article}
\usepackage[english,brazilian]{babel}
\usepackage[alf]{abntex2cite}
\usepackage[utf8]{inputenc}
\usepackage[T1]{fontenc}
\usepackage[top=25mm, bottom=20mm, left=20mm, right=20mm]{geometry}
\usepackage{framed}
\usepackage{booktabs}
\usepackage{color}
\usepackage{hyperref}
\usepackage{graphicx}
\usepackage{amsfonts}
\usepackage{subfigure}
\usepackage{enumerate}
\usepackage{float}
\graphicspath{{./Figuras/}}    
\definecolor{shadecolor}{rgb}{0.8,0.8,0.8}
\pagenumbering{arabic}

\usepackage{pgf,tikz}
\usetikzlibrary{arrows}

\usepackage{multicol}
\setlength{\columnseprule}{0pt}

%Cabeçario
\usepackage{fancyhdr}
\lhead{\rightmark}
\renewcommand{\footrulewidth}{0.5pt}

\setlength{\parindent}{1.25cm}%paragrafo

\pagestyle{fancy}
\def\MakeUppercase{} %Fonte minúscula no cabeçario

\pagenumbering{arabic}

\usepackage{amsthm}
\newtheorem{exem}{Exemplo}[section]

\usepackage[skip=10pt]{caption}
\captionsetup{font={stretch=0.4,small}} 

\newcommand{\m}[1]{\({#1}\)}
\newcommand{\M}[1]{\[{#1}\]}
\newcommand{\sol}{\noindent \textbf{Solução }}

\begin{document}
\begin{enumerate}
\item Qual o seno do maior ângulo de um triângulo cujos lados medem 4, 6 e 8 metros.
\begin{enumerate}
\item \(\frac{\sqrt{15}}{4}\)
\item \(\frac{1}{4}\)
\item \(\frac{1}{2}\)
\item \(\frac{\sqrt{10}}{4}\)
\item \(\frac{\sqrt{3}}{2}\)
\end{enumerate}

\item Um terreno de forma triangular tem frente de 10 m e 20 m, em ruas que formam, entre si, um ângulo de 120º. A medida do terceiro lado do terreno, em metros, é:
\begin{enumerate}
\item \(10\sqrt{5}\)
\item \(10\sqrt{7}\)
\item \(10\sqrt{6}\)
\item \(26\)
\item \(20\sqrt{2}\)
\end{enumerate}

\end{enumerate}
\end{document}