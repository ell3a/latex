\documentclass[oneside,a4paper,12pt]{article}
\input preambulo.tex

\begin{document}

\section{Exercícios}

\begin{enumerate}

\item A Tabela~\ref{tabela1} a seguir traz dados sobre a taxa de homicídios a cada 100 mil pessoas por ano em algumas unidades federativas dos Estados Unidos.

\begin{table}[htb]
\center
\begin{tabular}{p{1cm}lrc}
\hline 
 & \textbf{Estado} & \textbf{População} & \textbf{Taxa de Homicídio} \\ 
\hline 
1 & Alabama & 4.779.736 & 5,7 \\ 

2 & Alasca & 710.231 & 5,6 \\ 
 
3 & Arizona & 6.392.017 & 4,7 \\ 
 
4 & Arkansas & 2.915.918 & 5,6 \\ 

5 & Califórnia & 37.253.956 & 4,4 \\ 

6 & Colorado & 5.029.196 & 2,8 \\ 

7 & Connecticut & 3.574.097 & 2,4 \\ 

8 & Delaware & 897.934 & 5,8 \\ 
\hline
\end{tabular} 
\caption{} 
\label{tabela1}
\end{table}

Com base nas informações apresentadas na questão e na tabela anterior responda aos seguintes itens:
\begin{enumerate}
\item Qual a média da população para os estados observados?
\item Qual da taxa de homicídios?
\item Qual a moda e a mediana das taxas de homicídio observadas?
\item Complete a tabela a seguir com o número de homicídios?

\begin{table}[htb]
\center
\begin{tabular}{p{1cm}lc}
\hline 
 & \textbf{Estado} & \textbf{Número de Homicídio} \\ 
\hline 
1 & Alabama &  \\ 

2 & Alasca &  \\ 
 
3 & Arizona &  \\ 
 
4 & Arkansas &  \\ 

5 & Califórnia &  \\ 

6 & Colorado &  \\ 

7 & Connecticut &  \\ 

8 & Delaware &  \\ 
\hline
\end{tabular} 
\caption{} 
\end{table}

\item Qual média do número de homicídios para os estados observados?

\end{enumerate}

\end{enumerate}
\end{document}