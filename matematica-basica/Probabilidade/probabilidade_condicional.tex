\documentclass[twocolumn,oneside,a4paper,12pt]{article}

\usepackage[english,brazilian]{babel}
\usepackage[alf]{abntex2cite}
\usepackage[utf8]{inputenc}
\usepackage[T1]{fontenc}
\usepackage[top=25mm, bottom=20mm, left=20mm, right=20mm]{geometry}
\usepackage{framed}
\usepackage{booktabs}
\usepackage{color}
\usepackage{hyperref}
\usepackage{graphicx}
\usepackage{amsfonts}
\usepackage{subfigure}
\usepackage{enumerate}
\usepackage{float}
\graphicspath{{./Figuras/}}    
\definecolor{shadecolor}{rgb}{0.8,0.8,0.8}
\pagenumbering{arabic}

\usepackage{pgf,tikz}
\usetikzlibrary{arrows}

\usepackage{multicol}
\setlength{\columnseprule}{0pt}

%Cabeçario
\usepackage{fancyhdr}
\lhead{\rightmark}
\renewcommand{\footrulewidth}{0.5pt}

\setlength{\parindent}{1.25cm}%paragrafo

\pagestyle{fancy}
\def\MakeUppercase{} %Fonte minúscula no cabeçario

\pagenumbering{arabic}

\usepackage{amsthm}
\newtheorem{exem}{Exemplo}[section]

\usepackage[skip=10pt]{caption}
\captionsetup{font={stretch=0.4,small}} 

\newcommand{\m}[1]{\({#1}\)}
\newcommand{\M}[1]{\[{#1}\]}
\newcommand{\sol}{\noindent \textbf{Solução }}

\begin{document}
\pagestyle{empty}
\cabecalho

\begin{enumerate}
\item Na tabela a seguir tem os números sobre os funcionários de uma fábrica no setores de produção e distribuição:

\begin{tabular}{cccc}
\hline
 & \textbf{Produção} & \textbf{Distribuição} \\ 
\hline
\textbf{Homens} & 248 & 62 \\ 
\textbf{Mulheres} & 192 & 11 \\ 
\hline
\end{tabular} 

Sorteando um funcionário desses dois setores, calcule a probabilidade de:

\begin{enumerate}
\item Ser homem:
\item Ser mulher:
\item Ser homem, sabendo que o funcionário é do setor de distribuição:
\item Ser do setor de produção, sabendo que é uma mulher
\item Ser uma mulher do setor de produção:
\item Ser um homem do setor de distribuição:
\end{enumerate}

\item Em uma urna há 12 bolas de mesmo tamanho e textura, sendo 7 pretas e 5 brancas. Tirando sucessivamente e sem reposição duas dessas bolas, qual evento tem maior probabilidade: sair duas bolas de mesma cor ou sair duas bolas de cores distintas?

\item Em um escritório, há dois porta-lápis: o porta-lápis A com 12 lápis, dentro os quais 5 estão apontados, e o porta-lápis B com 15 lápis, dentre os quais 8 estão apontados. Um funcionários retira ao acaso um lápis do porta-lápis A e o coloca no porta-lápis B. Novamente, ao acaso, ele retira um lápis qualquer do porta-lápis B. Qual a probabilidade que este último lápis não tenha ponta? 

\item Um baralho consiste em 100 cartões numerados de 1 a 100. Retiram-se dois cartões ao acaso (sem reposição). A probabilidade de que a soma dos dois números seja igual a 100 é:
\begin{enumerate}
\item \(\frac{49}{4950}\)
\item \(\frac{50}{4950}\)
\item \(\frac{49}{5000}\)
\item \(\frac{50}{4851}\)
\item \(\frac{51}{4851}\)
\end{enumerate}

\item A probabilidade de um nadador A queimar a largada em uma competição é de 18\%; para o nadador B essa probabilidade é de 12\%. Se os dois nadadores estão disputando uma prova, qual é a probabilidade de que:   
\begin{enumerate}
\item ambos queimem a largada?
\item nenhum deles queime a largada?
\item pelo menos um queime a largada? 
\end{enumerate}

\item Dois jogadores disputam uma série de rodadas de cara-ou-coroa. No início, cada jogador dispõe de duas fichas. A cada rodada o vencedor ganha uma ficha do perdedor. O jogo termina quando um dos jogadores fica sem fichas. Determine a probabilidade de.
\begin{enumerate}
\item O jogo acabar com 3 rodadas?
\item Tenha pelo menos 8 rodadas?
\item Acabe na décima rodada? 
\end{enumerate}

\item Pedro e José jogam um dado não tendencioso. Se o resultado for 6, pedro vence, se for 1 ou 2, José vence, em qualquer outro caso, jogam novamente até que haja um vencedor. Qual é a probabilidade que esse vencedor seja Pedro?

(Dica: P.G. razão 1/6)

%-------------------------------------------------------------

\item Na tabela a seguir tem os números sobre os funcionários de uma fábrica no setores de produção e distribuição:

\begin{tabular}{cccc}
\hline
 & \textbf{Produção} & \textbf{Distribuição} \\ 
\hline
\textbf{Homens} & 452 & 52 \\ 
\textbf{Mulheres} & 280 & 64 \\ 
\hline
\end{tabular} 

Sorteando um funcionário desses dois setores, calcule a probabilidade de:

\begin{enumerate}
\item Ser homem, sabendo que o funcionário é do setor de distribuição:
\item Ser do setor de produção, sabendo que é uma mulher
\end{enumerate}

\item Em uma urna há 20 bolas de mesmo tamanho e textura, sendo 12 pretas e 8 brancas. Tirando sucessivamente e sem reposição duas dessas bolas, qual a probabilidade de
\begin{enumerate}
\item sair duas bolas de mesma cor:
\item sair duas bolas de cores distintas:
\end{enumerate}

\item Em um escritório, há dois porta-lápis: o porta-lápis A com 10 lápis, dentro os quais 3 estão apontados, e o porta-lápis B com 9 lápis, dentre os quais 4 estão apontados. Um funcionários retira ao acaso um lápis do porta-lápis A e o coloca no porta-lápis B. Novamente, ao acaso, ele retira um lápis qualquer do porta-lápis B. Qual a probabilidade que este último lápis não tenha ponta? 

\begin{enumerate}
\item \(0,64\)
\item \(0,57\)
\item \(0,052\)
\item \(0,42\)
\item \(0,50\)
\end{enumerate}

\item Dois jogadores disputam uma série de rodadas de cara-ou-coroa. No início, cada jogador dispõe de duas fichas. A cada rodada o vencedor ganha uma ficha do perdedor. O jogo termina quando um dos jogadores fica sem fichas. Determine a probabilidade de.
\begin{enumerate}
\item O jogo acabar com 3 rodadas?
\item Tenha pelo menos 8 rodadas?
\item Acabe na décima rodada? 
\end{enumerate}

\item A probabilidade de um nadador A queimar a largada em uma competição é de 12\%; para o nadador B essa probabilidade é de 15\%. Se os dois nadadores estão disputando uma prova, qual é a probabilidade de que:   
\begin{enumerate}
\item ambos queimem a largada?
\item nenhum deles queime a largada?
\item pelo menos um queime a largada? 
\end{enumerate}

\end{enumerate}

\FRASE
\end{document}
