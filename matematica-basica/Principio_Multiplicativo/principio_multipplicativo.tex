\documentclass[twocolumn,oneside,a4paper,12pt]{article}

\usepackage[english,brazilian]{babel}
\usepackage[alf]{abntex2cite}
\usepackage[utf8]{inputenc}
\usepackage[T1]{fontenc}
\usepackage[top=25mm, bottom=20mm, left=20mm, right=20mm]{geometry}
\usepackage{framed}
\usepackage{booktabs}
\usepackage{color}
\usepackage{hyperref}
\usepackage{graphicx}
\usepackage{amsfonts}
\usepackage{subfigure}
\usepackage{enumerate}
\usepackage{float}
\graphicspath{{./Figuras/}}    
\definecolor{shadecolor}{rgb}{0.8,0.8,0.8}
\pagenumbering{arabic}

\usepackage{pgf,tikz}
\usetikzlibrary{arrows}

\usepackage{multicol}
\setlength{\columnseprule}{0pt}

%Cabeçario
\usepackage{fancyhdr}
\lhead{\rightmark}
\renewcommand{\footrulewidth}{0.5pt}

\setlength{\parindent}{1.25cm}%paragrafo

\pagestyle{fancy}
\def\MakeUppercase{} %Fonte minúscula no cabeçario

\pagenumbering{arabic}

\usepackage{amsthm}
\newtheorem{exem}{Exemplo}[section]

\usepackage[skip=10pt]{caption}
\captionsetup{font={stretch=0.4,small}} 

\newcommand{\m}[1]{\({#1}\)}
\newcommand{\M}[1]{\[{#1}\]}
\newcommand{\sol}{\noindent \textbf{Solução }}

\begin{document}
\maketitle

\section{Introdução}

No o desenho a seguir esquematiza as ligações, por meio de estradas, entre três cidades: \(A\), \(B\) e \(C\).

	\begin{figure}[!tbh]
	\center
	\includegraphics[width=6cm]{00}
	\end{figure}

\begin{enumerate}[(a)]
\item De quantas formas podemos ir de \(A\) até \(C\)?
\item De quantas formas é possível ir de \(A\) até \(C\), e depois de voltar à cidade \(A\) sem repetir caminhos?
\end{enumerate}

\section{O Princípio}
 Se ha \(x\) modos de tomar uma decisão \(D_1\) e, tomada a decisão \(D_1\), há \(y\) modos de tomar a decisão \(D_2\), então o número de modos de tomar sucessivamente as decisões \(D_1\) e \(D_2\) é o produto \(x \cdot y\).
 
\begin{exemplo}
Um rapaz tem 5 camisetas de cores distintas, e 3 bermudas diferentes. De quantas formas ele pode se vestir escolhendo uma camiseta e uma bermuda: 
\end{exemplo}

\begin{exemplo}
A figura a seguir esquematiza a ligação entre três cidades por meio de estradas.

	\begin{figure}[!tbh]
	\center
	\includegraphics[width=6cm]{01}
	\end{figure}

\noindent As setas pretas representam as estradas asfaltadas, e as setas em vermelho as estradas que não são asfaltadas. Com base nessas informações responda aos seguintes itens:
\end{exemplo}

\begin{enumerate}[(a)]
\item De quantas formas podemos ir de \(A\) até \(C\)?
\item De quantas formas é possível ir de \(A\) até \(C\), e depois voltar à cidade \(A\) sem repetir caminhos?
\item De quantas formas é possível ir de \(A\) até \(C\) passando por uma estrada asfaltada e por uma não asfaltada?
\item Uma competição de motocross passará somente pelas as estradas não asfaltadas entre as cidades \(A\) e \(C\), de quantas formas esse trajeto pode ser feito?
\end{enumerate}

\begin{exemplo}
Na figura a seguir temos a representação de uma bandeira composta por três áreas que devem ser pintadas com um de seis cores disponíveis:
	
	\begin{figure}[!tbh]
	\center
	\includegraphics[width=4cm]{02}
	\end{figure}
\end{exemplo}

\begin{enumerate}[(a)]
\item Sem repetir cores, de quantas formas é possível pintar a bandeira?
\item Podendo repetir cores na pintura da bandeira, mas sem que áreas vizinhas tenham a mesma cor, de quantas formas é possível pintá-la?
\end{enumerate}

\begin{exemplo}
Um sistema de senhas de uma empresa aceita senhas feitas com os algarismos 0, 1, 2, 3, 4 e 5. E todas as senhas devem ser compostas por 4 algarismos, sendo que não devem começar por zero. Sabendo que não pode haver repetição de algarismos na senha, qual a quantidades de maneiras diferentes de montar a senha?
\end{exemplo}

\begin{exemplo}
O professor de artes de uma escola também é um apaixonado por matemática. Ele deseja saber de quantas formas é possível colorir a figura, com 8 cores distintas, de forma que setores da figura que tenham arestas em comum não tenham a mesma cor.
	
	\begin{figure}[!tbh]
	\center
	\includegraphics[width=4cm]{03}
	\end{figure}

\noindent De quantas maneiras essa figura pode ser colorida?
\end{exemplo}

\section{Exercicios}

 Página 230: 1, 2, 3, 4

\section{Exemplos de Probabilidade}
\begin{exemplo}
E uma urna estão fichas, cada uma contendo uma das formas de pintar a figura representada a seguir, com uma das cores: azul, verde, rosa e vermelho, de modo que não haja repetição de cores na na pintura das faixas.
	
	\begin{figure}[!tbh]
	\center
	\includegraphics[width=4cm]{02}
	\end{figure}

\noindent Ao tirar uma das fichas, qual a probabilidade de que a mesma tenha uma das faixas pintada de vermelho.
\end{exemplo}


\begin{exemplo}
Em uma urna estão fichas cada uma com um dos números de três dígitos que pode ser formados com os algarismos 0, 1, 2, 3 e 4, retirando-se uma das fichas ao acaso, qual a probabilidade de que o seu número seja par?
\end{exemplo}

\end{document}
