\documentclass[twocolumn,oneside,a4paper,12pt]{article}
\usepackage[english,brazilian]{babel}
\usepackage[alf]{abntex2cite}
\usepackage[utf8]{inputenc}
\usepackage[T1]{fontenc}
\usepackage[top=25mm, bottom=20mm, left=20mm, right=20mm]{geometry}
\usepackage{framed}
\usepackage{booktabs}
\usepackage{color}
\usepackage{hyperref}
\usepackage{graphicx}
\usepackage{amsfonts}
\usepackage{subfigure}
\usepackage{enumerate}
\usepackage{float}
\graphicspath{{./Figuras/}}    
\definecolor{shadecolor}{rgb}{0.8,0.8,0.8}
\pagenumbering{arabic}

\usepackage{pgf,tikz}
\usetikzlibrary{arrows}

\usepackage{multicol}
\setlength{\columnseprule}{0pt}

%Cabeçario
\usepackage{fancyhdr}
\lhead{\rightmark}
\renewcommand{\footrulewidth}{0.5pt}

\setlength{\parindent}{1.25cm}%paragrafo

\pagestyle{fancy}
\def\MakeUppercase{} %Fonte minúscula no cabeçario

\pagenumbering{arabic}

\usepackage{amsthm}
\newtheorem{exem}{Exemplo}[section]

\usepackage[skip=10pt]{caption}
\captionsetup{font={stretch=0.4,small}} 

\newcommand{\m}[1]{\({#1}\)}
\newcommand{\M}[1]{\[{#1}\]}
\newcommand{\sol}{\noindent \textbf{Solução }}

\begin{document}
\pagestyle{empty}
\begin{enumerate}
\item O topo de um farol \m{F} é visto de um ponto \m{A} sob um ângulo \m{\alpha = 25^{\circ}}, conforme a figura a seguir:

\figura{01}{8cm}

Sabendo que \m{AC=80} m, calcule:
\begin{enumerate}
\item A medida de \m{AC}, altura do farol:
\item A distância entre o ponto \m{A} e o ponto \m{B}:
\end{enumerate}

\item Do topo de um farol é possível ver um possível ver uma casinha de cachorro, conforme a figura a seguir:

\figura{02}{8cm}

Sabendo que a distância entre o topo do farol e casinha de cachorro é 120 m e que \m{\alpha = 35^{\circ}}, calcule:
\begin{enumerate}
\item A altura do farol:
\item A distância entre o farol e a casinha:
\end{enumerate}

\item De uma casa é possível ver o topo de uma torre sob um ângulo \m{\alpha = 15^{\circ}}, conforme a figura a seguir.

\figura{03}{8cm}

A distância entre a torre e casa é de 500 m. Calcule a altura da torre, e a distância entre a torre a casa:

\item Para medir a altura de uma montanha, um geólogo observou o ponto mais alto da mesma \m{A} do chão em ponto \m{Q} sob um ângulo de \m{35^{\circ}}, em seguida ele observou o topo da montanha do ponto \m{P}, distante 300 m de \m{Q}, sob um ângulo de \m{25^{\circ}}.

\figura{04}{9cm}

Qual a altura da montanha obtida pelo geólogo.

\item De um ponto \m{A} é possível ver o ponto mais alto \m{P} de uma montanha sob um ângulo \m{\alpha}. Do ponto \m{B}, distante 250 m de \m{A}, é possível observar \m{P} sob um ângulo \m{\beta}. Conforme a figura a seguir

\figura{05}{9cm}

Sabendo \m{\alpha = 10^{\circ}} e que \m{\beta = 20^{\circ}}, calcule a altura da montanha.

\item O ponto mais alto \m{M} de uma montanha pode ser observado de um ponto \m{P}, sob um ângulo \m{\alpha=12^{\circ}}, e pode ser observado de um ponto \m{Q}, distante 420 m de \m{P}, sob um ângulo \m{\beta=18^{\circ}}. 

\figura{06}{9cm}

Com base nas informações disponibilizadas na questão calcule a altura da montanha.

\item Para medir a altura de uma montanha um engenheiro observou o ponto mais alto da mesma em dois pontos. Conforme a figura a seguir:

\newpage
\figura{07}{8cm}

No ponto \m{A} é possível observar o topo da montanha sob um ângulo \m{30^{\circ}}, e do ponto \m{B} é possível ver o topo da montanha sob um ângulo de \m{60^{\circ}}. Sabendo que a distância entre \m{A} e \m{B} é 1600 m, calcule a altura da montanha:

\item Na figura a seguir um engenheiro usa um teodolito para medir os ângulos de observação do ponto mais alto de uma árvore:

\figura{08}{8cm}

Sabendo que a distância entre os pontos \m{A} e \m{B} é de 7,5 m. Calcule a distância entre os pontos \m{M} e \m{N}.

\item Na figura a seguir a distância entre os pontos \m{A} e \m{B} é 12 m.

\figura{09}{8cm}

Calcule a distância entre o ponto \m{M}, ponto mais alto da árvore e o chão:

\item No mapa a seguir a distância entre os pontos \m{B} e \m{C} é 800 m.

\figura{10}{7cm}

Sabendo que \m{\beta = 31^{\circ}}, calcule a distância entre o ponto \m{A} e \m{B}.

\item Para calcular a distância entre os pontos \m{C} e \m{A} em um mapa. Um topógrafo, andou até o ponto \m{B}, distante 2500 m de \m{C}, em seguida mediu o ângulo de observação do ponto \m{A}, em relação ao segmento \m{BC}, como na figura a seguir:

\figura{11}{7cm}

Sabendo que \m{\beta = 58^{\circ}}, calcule a distância entre os pontos \m{C} e \m{A}.

\item Huata, Huancané e Moho são cidades peruanas, elas formam entre si um triângulo conforme a figura a seguir:

\figura{12}{8cm}

Sabendo que a distância em linha reta entre Huata e Huancané é de 51 km, calcule a distância entre Huata e Moho e a distância entre Huancané e Moho.

\item O ponto mais alto de uma montanha russa pode ser visto sob dois ângulos conforme a figura a seguir.

\figura{13}{9cm}

Sabendo que os ângulos \m{\alpha} e \m{\beta} medem respectivamente \m{35^{\circ}} e \m{65^{\circ}}, calcule a altura da montanha russa.

\item De um ponto \m{B} é possível observar o ponto mais alto de uma casa sob um ângulo \m{\alpha = 16^{\circ}} e o ponto mais alto de uma catedral sob um ângulo \m{\beta = 76^{\circ}}, conforme a figura a seguir:

\figura{14}{9cm}

Sabendo que a altura da casa é 4 m, e que a altura da catedral é 30 m. Calcule a distância entre os pontos \m{M} e \m{N}:

\item Dois cabos são usados para sustentar uma torre conforme a figura a seguir

\figura{x01}{9cm}

Sabendo que a distância entre os pontos M e N deve ser de 38 m. Calcule o comprimento dos cabos, uma vez que a largura CD = 16 m.

\item Dois cabos são usados para sustentar uma caixa de água conforme a figura a seguir

\figura{x02}{9cm}

Sabendo que a distância entre os pontos M e N deve ser de 8 m. Calcule o comprimento dos cabos, uma vez que a largura CD = 2 m.

\item Duas caixas de água serão interligadas conforme a figura a seguir

\figura{x03}{9cm}

Sabendo que a segmento \m{AC} tem comprimento 20 m. Calcue o comprimento do cano \m{AB}, sabendo que o ângulo \m{B\widehat{A}C = 20^{\circ}}

\item Um foguete é lançado de uma base que faz um ângulo \m{\alpha = 48^{\circ}} com a horizontal, conforme a figura a seguir.

\figura{x05}{5cm}

Sabendo que no ar o foguete percorreu no ar 250 m, calcule a distância percorrida no solo:

\item Um avião decola em trajetória que forma um ângulo de \m{\alpha = 15^{\circ}}, conforme a figura a seguir.

\figura{x04}{9cm}

Sabendo que a altura que o avião está é 2000 m, calcule a distância percorrida pelo mesmo no ar.

\item Um foguete é lançado sob um ângulo \m{\alpha = 80^{\circ}}.

\figura{x06}{5cm}

Sabendo que o foguete percorreu 800 m no ar. Calcule a que altura o mesmo está:

\item O ponto mais alto de um prédio pode ser visto sob dois ângulos conforme a figura a seguir.

\figura{x07}{9cm}

Sabendo que os ângulos \m{\alpha} e \m{\beta} medem respectivamente \m{10^{\circ}} e \m{20^{\circ}}, calcule a altura do prédio, sabendo que a distância entre os pontos \m{P} e \m{Q} é 30 m.

\item O ponto mais alto de um prédio pode ser visto sob dois ângulos conforme a figura a seguir.

\figura{x08}{9cm}

Sabendo que os ângulos \m{\alpha} e \m{\beta} medem respectivamente \m{16^{\circ}} e \m{50^{\circ}}, calcule a altura do prédio, sabendo que a distância entre os pontos \m{P} e \m{Q} é 25 m.

\item Um fio será ligado de uma torre de transmissão no ponto \m{P} até um poste no ponto \m{A}, conforme a figura a seguir:

\figura{x09}{9cm}

Sabendo que altura do poste (distância do ponto \m{B} até o ponto \m{Q}) é 5,0 m, e que os ângulos \m{\alpha} e \m{\beta} medem respectivamente \m{12^{\circ}} e \m{74^{\circ}}, calcule o comprimento do fio \m{AB}.

\end{enumerate}

\begin{flushright}
\textit{O homem é livre para fazer o que quer,\\mas não para querer o que quer.}
\end{flushright}
\end{document}