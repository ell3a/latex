\documentclass[twocolumn,oneside,a4paper,12pt]{article}
\input preambulo.tex

\begin{document}
\maketitle
\FRASE
	
%	\begin{figure}[!ht]
%	\center
%	\includegraphics[width=8cm]{dado-moeda}
%	\end{figure}

\section*{Naturais}
O conjunto dos números naturais \m{\N} é formado pelos números usados na contagem:

\M{\N = \{1,2,3,4,5,\ldots\}}

Os números naturais tem algumas propriedades que podemos listar a seguir:
\begin{enumerate}[i.]
\item Possui um menor elemento: 1;
\item Não possui um elemento máximo;
\item Todo elemento de \m{\N} tem um sucessor em \m{\N};
\item Se \m{n \in \N} e \m{n \neq 1}, \m{n} tem um antecessor em \m{\N}.
\end{enumerate}

\section*{Inteiros}
O conjunto dos números inteiros \m{\Z} é formado por todos os números naturais, seus inversos aditivos, e o zero:
\M{\Z = \{\ldots ,-4,-3,-2,-1,0,1,2,3,4,\ldots\}}
O conjunto \m{\Z} tem algumas propriedades:
\begin{enumerate}[i.]
\item Não possui um elemento mínimo nem um elemento máximo;
\item Todo elemento de \m{\Z} tem um sucessor e antecessor em \m{\Z}.
\end{enumerate}

\section*{Racionais}
O conjunto dos números racionais \m{\Q} é formado por todos os números que podem ser escritos na forma de fração:
\M{\Q = \left\lbrace \frac{p}{q}; p,q \in \Z \mbox{ e } q \neq 0 \right\rbrace}

Todos os naturais são inteiros, assim dizemos que o conjunto dos números naturais é um subconjunto dos números inteiros:
\M{\N \subset \Z.}
\noindent Da mesma forma, todo número inteiro é também um número racional
\M{\Z \subset \Q.}

\subsection{Frações}
Dadas duas frações 

\subsection*{Arrendodamentos}

\subsection*{Notação Científica}


\subsection*{Dízimas Periódicas}
As dízimas periódicas são números racionais, assim podemos transformar uma dízima períodica em uma fração:

\section*{Irracionais}


\section*{Reais}

\end{document}