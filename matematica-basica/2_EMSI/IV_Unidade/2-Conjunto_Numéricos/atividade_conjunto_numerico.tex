\documentclass[twocolumn,oneside,a4paper,11pt]{article}
\usepackage[english,brazilian]{babel}
\usepackage[alf]{abntex2cite}
\usepackage[utf8]{inputenc}
\usepackage[T1]{fontenc}
\usepackage[top=10mm, bottom=10mm, left=10mm, right=10mm]{geometry}
\usepackage{framed}
\usepackage{booktabs}
\usepackage{color}
\usepackage{hyperref}
\usepackage{graphicx}
\usepackage{float}
\usepackage{amssymb} %Símbolos de Conjuntos numéricos
\usepackage{multicol} %Várias Colunas
\graphicspath{{./Figuras/}}    
\definecolor{shadecolor}{rgb}{0.8,0.8,0.8}

\newcommand{\EREM}{EREM Regina Pacis}
\newcommand{\curso}{EMSI}
\newcommand{\professor}{Prof. Leandro Vieira}

\newcommand{\PR}[1]{\ensuremath{\left[#1\right]}} %Comandos para parênteses grandes
\newcommand{\PC}[1]{\ensuremath{\left(#1\right)}}
\newcommand{\chav}[1]{\ensuremath{\left\{#1\right\}}}

\newcommand{\m}[1]{\(\displaystyle #1\)}
\newcommand{\M}[1]{\[#1\]}

\begin{document}
\pagestyle{empty}
\noindent \textbf{conjunto numéricos}

	\begin{center}
		\EREM
		\par %pula uma linha
		\curso
		\par
		\professor
		\par
		\vspace{10pt}
		\textbf{\large{Atividade de Matemática da IV Unidade}}
	\end{center}
	
\begin{enumerate}
\item Faça as seguintes operações:
\begin{multicols}{3}
	\begin{enumerate}
	\item \m{\frac{5}{6} + \frac{8}{5}}
	\item \m{\frac{1}{2} + \frac{2}{3}}
	\item \m{\frac{4}{9} + \frac{1}{7}}
	\item \m{\frac{7}{8} - \frac{1}{2}}
	\item \m{\frac{9}{5} - \frac{3}{4}}
	\item \m{\frac{6}{7} - \frac{2}{5}}
	\item \m{2 + \frac{1}{3}}
	\item \m{3 - \frac{2}{5}}
	\item \m{1 + \frac{1}{2}}
	\end{enumerate}
\end{multicols}

\item Em uma turma com 75 estundantes, 2/3 é composta por mulheres, qual a quantidade de homens dessa turma:

\item Ao receber um pagamento de R\$ 640,00, um comerciamente usou 3/8 do valor recebido para pagar algumas despesas da loja. Quanto o comerciamente usou?

\item O gerente de um loja percebeu que em um mês o faturamento foi de R\$ 8.000,00. Mas ele calculou que no mês seguinte esse valor tem que ser 1/8 maior, para que a loja não tenha prejuízo, qual deve ser o faturamento no mês seguinte? 

\item Escreva as seguintes dízimas como uma fração como númerador e denominador inteiros:
\begin{multicols}{2}
	\begin{enumerate}
	\item \m{2,3333 \ldots}
	\item \m{6,7777 \ldots}
	\item \m{11,1111 \ldots} 
	\item \m{5,1222 \ldots}
	\item \m{5,3999 \ldots} 
	\item \m{4,8282 \ldots}
	\item \m{4,0808 \ldots} 
	\item \m{9,412412 \ldots}
	\item \m{10,385385 \ldots} 
	\end{enumerate}
\end{multicols}

						
\end{enumerate}
	
\flushbottom
\flushright
``\textit{Quem nunca errou nunca experimentou nada novo.}''\\\textbf{Albert Einstein}
	
\end{document}