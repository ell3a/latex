\documentclass[twocolumn,oneside,a4paper,12pt]{article}
\usepackage[english,brazilian]{babel}
\usepackage[alf]{abntex2cite}
\usepackage[utf8]{inputenc}
\usepackage[T1]{fontenc}
\usepackage[top=25mm, bottom=20mm, left=20mm, right=20mm]{geometry}
\usepackage{framed}
\usepackage{booktabs}
\usepackage{color}
\usepackage{hyperref}
\usepackage{graphicx}
\usepackage{amsfonts}
\usepackage{subfigure}
\usepackage{enumerate}
\usepackage{float}
\graphicspath{{./Figuras/}}    
\definecolor{shadecolor}{rgb}{0.8,0.8,0.8}
\pagenumbering{arabic}

\usepackage{pgf,tikz}
\usetikzlibrary{arrows}

\usepackage{multicol}
\setlength{\columnseprule}{0pt}

%Cabeçario
\usepackage{fancyhdr}
\lhead{\rightmark}
\renewcommand{\footrulewidth}{0.5pt}

\setlength{\parindent}{1.25cm}%paragrafo

\pagestyle{fancy}
\def\MakeUppercase{} %Fonte minúscula no cabeçario

\pagenumbering{arabic}

\usepackage{amsthm}
\newtheorem{exem}{Exemplo}[section]

\usepackage[skip=10pt]{caption}
\captionsetup{font={stretch=0.4,small}} 

\newcommand{\m}[1]{\({#1}\)}
\newcommand{\M}[1]{\[{#1}\]}
\newcommand{\sol}{\noindent \textbf{Solução }}

\begin{document}
\begin{enumerate}
\item Calcule o volume do paralelepípedo a seguir:
\figura{01}{8cm}

\item A figura seguir é um tanque.
\figura{02}{8cm}

Calcule o volume desse tanque:

\item Em um clube há uma piscina no forma da figura a seguir.
\figura{03}{8cm}

Qual o volume dessa piscina.

\item Em uma fábrica há uma caixa de água com o formato da figura a seguir
\figura{04}{5cm}

Calcule o volume do caixa de água:

\item Qual é o volume do prisma da imagem a seguir, sabendo que ele é um prisma reto e sua base é quadrada?
\figura{05}{5cm}

\item Considere um reservatório, em forma de paralelepípedo retângulo, cujas medidas são 8 m de comprimento, 5 m de largura e 1,2 m de profundidade. Bombeia-se água para dentro desse reservatório, inicialmente vazio, a uma taxa de 2 litros por segundo. Com base nessas informações, é CORRETO afirmar que, para se encher completamente esse reservatório, serão necessários:
\begin{enumerate}
\item 240 min .
\item 400 min .
\item 480 min .
\item 40 min .
\end{enumerate}

\item O volume de uma piscina em forma de prisma de base quadrada é 3125 metros cúbicos. Sabendo que a altura dessa piscina é de 5 metros cúbicos, qual é a medida da aresta de sua base em metros?
\begin{enumerate}
\item 5 m
\item 10 m
\item 15 m
\item 20 m
\item 25 m
\end{enumerate}

\item Um bloco retangular possui como base um retângulo com área de 120 cm\m{^2}. Sabendo que o volume desse bloco é de 480 cm\m{^3}, qual é sua altura em centímetros?
\begin{enumerate}
\item 4
\item 5
\item 6
\item 7
\item 8
\end{enumerate}

\item Um estoquista, ao conferir a quantidade de determinado produto embalado em caixas cúbicas de arestas medindo 40 cm, verificou que o estoque do produto estava empilhado de acordo com a figura que segue:
\figura{06}{4cm}

Ao realizar corretamente os cálculos do volume dessa pilha de caixas, o resultado obtido foi:
\begin{enumerate}
\item 0,64 m\m{^3}
\item 1,6 m\m{^3}
\item 6,4 m\m{^3}
\item 16 m\m{^3}
\item 64 m\m{^3}
\end{enumerate}

\item Um aquário possui o formato de um paralelepípedo com as seguintes dimensões:
\figura{07}{4cm}

Determine quantos litros de água são necessários para encher o aquário.

\item O degrau de uma escada lembra a forma de um paralelepípedo com as seguintes dimensões: 1 m de comprimento, 0,5 m de largura e 0,4 m de altura. Determine o volume total de concreto gasto na construção dessa escada sabendo que ela é constituída de 20 degraus.
\figura{08}{6cm}

\item Os papiros mostram que os egípcios antigos possuíam diversos conhecimentos matemáticos. Eles sabiam que o volume da pirâmide equivale a um terço do volume do prisma que a contém. A maior pirâmide egípcia, Quéops, construída por volta de 2560 a.C., tem uma altura aproximada de 140 metros e sua base é um quadrado com lados medindo aproximadamente 230 metros. Logo, o volume da pirâmide de Quéops é:



\end{enumerate}
\end{document}