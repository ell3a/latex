\documentclass[12pt,a4paper]{article}

\usepackage[english,brazilian]{babel}
\usepackage[alf]{abntex2cite}
\usepackage[utf8]{inputenc}
\usepackage[T1]{fontenc}
\usepackage[top=25mm, bottom=20mm, left=20mm, right=20mm]{geometry}
\usepackage{framed}
\usepackage{booktabs}
\usepackage{color}
\usepackage{hyperref}
\usepackage{graphicx}
\usepackage{amsfonts}
\usepackage{subfigure}
\usepackage{enumerate}
\usepackage{float}
\graphicspath{{./Figuras/}}    
\definecolor{shadecolor}{rgb}{0.8,0.8,0.8}
\pagenumbering{arabic}

\usepackage{pgf,tikz}
\usetikzlibrary{arrows}

\usepackage{multicol}
\setlength{\columnseprule}{0pt}

%Cabeçario
\usepackage{fancyhdr}
\lhead{\rightmark}
\renewcommand{\footrulewidth}{0.5pt}

\setlength{\parindent}{1.25cm}%paragrafo

\pagestyle{fancy}
\def\MakeUppercase{} %Fonte minúscula no cabeçario

\pagenumbering{arabic}

\usepackage{amsthm}
\newtheorem{exem}{Exemplo}[section]

\usepackage[skip=10pt]{caption}
\captionsetup{font={stretch=0.4,small}} 

\newcommand{\m}[1]{\({#1}\)}
\newcommand{\M}[1]{\[{#1}\]}
\newcommand{\sol}{\noindent \textbf{Solução }}

\begin{document}

\section{Definição}

Os poliedros podem ser definidos de várias formas, com maior ou menor generalidades. A definição de poliedro ao longo da história não foi unânime. Euler que enunciou o importante teorema que relaciona o número de vértices faces e arestas de um poliedro não deu uma definição para esses objetos.

Poliedro é a reunião de um número finito de polígonos planos, denominados faces, de modo que:

\begin{enumerate}[(a)]
\item Cada lado de um desses polígonos é também lado de um, e apenas um outro polígono. (Os lados dos polígonos que compõem o poliedro são denominados as arestas do poliedro)
\item A interseção de duas faces quaisquer ou é um lado comum, ou é um vértice, ou é vazia. (Os vértices dos polígonos que compõem o poliedro são denominados os vértices do poliedro)
\item ir, sobre o poliedro, de um ponto de uma face a um ponto de qualquer outra, sem passar por nenhum vértice (ou seja, cruzando apenas arestas).
\end{enumerate}

Condições para que três números naturais: \m{V}, \m{F} e \m{A} sejam respectivamente o número de vértices, faces e arestas de um poliedro
\begin{enumerate}[I -]
\item \m{F \geq 6};
\item \m{V-A+F = 2};
\item \m{A + 6 \leq 3F \leq 2A};
\item \m{A + 6 \leq 3V \leq 2A}.
\end{enumerate}

Famílias de poliedros primitivos
\begin{enumerate}[(i)]
\item \m{(4,6,4)}, tetraedro;
\item \m{(5,8,5)}, pirâmide de base quadrangular;
\item \m{(6,10,6)};
\end{enumerate}

Transformações de poliedros: \m{(2,3,1)} e \m{(1,3,2)}.
\M{(V,A,F) = (V',A',F')+x(2,3,1)+y(1,3,2)}

\begin{eqnarray}
d & = & \sqrt{4^2+4^2}\\
d & = & 4\sqrt{2} \\
x^2 + (2\sqrt{2})^2 & = & 4^2 \\
x^2 + 8 & = & 16 \\
x & = & 2\sqrt{2}
\end{eqnarray}

\newpage
\section{Geometria Plana}

\begin{exemplo}
Na Figura~\ref{quadrado1}, \m{ABCD} é um quadrado, a medida do ângulo \m{ABE} é \m{20^{\circ}} e
\m{BE = BC}. Qual é a medida do ângulo \m{DFC}?

\begin{figure}[!ht]
\center
\includegraphics[height=8cm]{Figuras/ex1}
\caption{}
\label{quadrado1}
\end{figure}


\end{exemplo}

%\begin{figure}[!ht]
%\center
%\subfigure[ref1][Tales]{\includegraphics[height=5cm]{Figuras/Tales_de_Mileto}}
%\qquad
%\qquad
%\subfigure[ref2][Localização de Mileto]{\includegraphics[height=5cm]{Figuras/Mileto}}
%\caption{}
%\end{figure}

\end{document}