\documentclass[12pt,a4paper]{book}
\usepackage[utf8]{inputenc}
\usepackage[portuguese]{babel}
\usepackage[T1]{fontenc}
\usepackage{amsmath}
\usepackage{amsfonts}
\usepackage{amssymb}
\usepackage{makeidx}
\usepackage{graphicx}
\usepackage{booktabs}
\usepackage{color}
\usepackage{hyperref}
\usepackage{float}
\graphicspath{{./Figuras/}}    
\usepackage{multicol}
\definecolor{shadecolor}{rgb}{0.8,0.8,0.8}
\usepackage{amssymb} %Símbolos de Conjuntos numéricos
\usepackage{multicol} %Várias Colunas
\usepackage[left=3cm,right=2cm,top=3cm,bottom=2cm]{geometry}
\author{Leandro Vieira}
\title{Álgebra Linear}

\begin{document}
\maketitle
\tableofcontents

\chapter{Considerações Iniciais}
\chapter{Transformções Lineares}
	\section{Núcleo de Uma transformação Linear}
	
	Se $T : V \longrightarrow W$ é uma transformação linear, então:
	
	$$N(T) = \ker(T) = \lbrace v \in V; T(v) = 0\rbrace$$
	
$N(T) \neq \emptyset$, pois $0 \in N(T)$.

	\subsection{Exemplo}

Seja $ T: {\mathbb{R}}^3  \longrightarrow {\mathbb{R}}^2 $ uma transformação linear onde $(x,y,z)\mapsto(x-y+4z, 3x+y+8z)$. Se
	
	
		\newpage \subsection{Problemas e Exercícios}
		

\end{document}
