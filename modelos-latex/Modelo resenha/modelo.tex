\documentclass[oneside,a4paper,12pt]{article}
\usepackage[top=3cm, bottom=2cm, left=3cm, right=2cm]{geometry}

% Aqui você coloca o título
\newcommand{\titulo}{Revisão de Matemática de matemática revisão revisão}
% Aqui você coloca o autor
\newcommand{\autor}{Leandro Vieira dos Santos}
% Aqui Você coloca o cabeçalho se presisar
\newcommand{\cabeca}{}

\usepackage[english,brazilian]{babel}
\usepackage[alf]{abntex2cite}
\usepackage[utf8]{inputenc}
\usepackage[T1]{fontenc}
\usepackage[top=25mm, bottom=20mm, left=20mm, right=20mm]{geometry}
\usepackage{framed}
\usepackage{booktabs}
\usepackage{color}
\usepackage{hyperref}
\usepackage{graphicx}
\usepackage{amsfonts}
\usepackage{subfigure}
\usepackage{enumerate}
\usepackage{float}
\graphicspath{{./Figuras/}}    
\definecolor{shadecolor}{rgb}{0.8,0.8,0.8}
\pagenumbering{arabic}

\usepackage{pgf,tikz}
\usetikzlibrary{arrows}

\usepackage{multicol}
\setlength{\columnseprule}{0pt}

%Cabeçario
\usepackage{fancyhdr}
\lhead{\rightmark}
\renewcommand{\footrulewidth}{0.5pt}

\setlength{\parindent}{1.25cm}%paragrafo

\pagestyle{fancy}
\def\MakeUppercase{} %Fonte minúscula no cabeçario

\pagenumbering{arabic}

\usepackage{amsthm}
\newtheorem{exem}{Exemplo}[section]

\usepackage[skip=10pt]{caption}
\captionsetup{font={stretch=0.4,small}} 

\newcommand{\m}[1]{\({#1}\)}
\newcommand{\M}[1]{\[{#1}\]}
\newcommand{\sol}{\noindent \textbf{Solução }}

\usepackage{helvet}
\renewcommand{\familydefault}{\sfdefault}

\begin{document}
\onehalfspacing

\maketitle

\begin{center} \fbox{ \begin{minipage}{14cm}\begin{center}
% Aqui você coloca a referência da obra
Nome da Obra    
\end{center} \end{minipage} } \end{center}

\vspace{5pt}

\section*{Primeiro Título}
\lipsum[1-2]

\section*{Segunto Título}
\lipsum[2-3]

\section*{Segundo Título}
\lipsum[4-5]

\section*{Terceiro Título}
\lipsum[6-7]

% Isso aqui serve para criar um espaçamento
% É só alterar o valor
\vspace{10pt}

% Aqui você coloca seus dados à esquerda
\begin{flushleft}
Universidade

Disciplina

nome
\end{flushleft}

% Aqui você coloca seus dados à esquerda
\begin{flushright}
    Universidade

    Disciplina
    
    nome        
\end{flushright}

\end{document}